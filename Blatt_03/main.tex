
\documentclass{article}

% Deactivate sectsty warning when loading sectsty {{{
\usepackage[immediate]{silence}
\WarningFilter[temp]{latex}{Command}
\usepackage{sectsty}
    \sectionfont{\normalfont\sffamily\bfseries\color{blue!40!black}}
    \subsectionfont{\normalfont\sffamily\bfseries\color{blue!30!black}}
\DeactivateWarningFilters[temp]
\makeatletter % disable the runtime redefinitions
\let\SS@makeulinesect\relax
\let\SS@makeulinepartchap\relax
\makeatother
% }}}

\usepackage[margin=4cm]{geometry}
    \setlength\parindent{0pt}
\usepackage{fancyhdr}
    \pagestyle{fancy}
\usepackage{fontspec}
    \setsansfont{Linux Biolinum O}
\usepackage{polyglossia}
    \setmainlanguage{english}
\usepackage{sectsty}
    \sectionfont{\normalfont\sffamily\bfseries\color{blue!40!black}}
    \subsectionfont{\normalfont\sffamily\bfseries\color{blue!30!black}}
\usepackage{pdfpages}
\usepackage{amsmath}
\usepackage{amssymb}
\usepackage{siunitx}
\usepackage{float}
\usepackage{booktabs}
\usepackage{subcaption}
\usepackage{graphicx}
\usepackage{xcolor}
\usepackage{listings}
    \lstset{language=Python,
	basicstyle=\footnotesize\ttfamily,
	breaklines=true,
	framextopmargin=50pt,
	frame=bottomline,
	backgroundcolor=\color{white!86!black},
	commentstyle=\color{blue},
	keywordstyle=\color{red},
	stringstyle=\color{orange!80!black}}
\usepackage{tikz}

\title{\textsf{\color{blue!40!black}3. Übung IBN}}
\author{Maurice Donner \and Ise Glade}

\begin{document}

\maketitle
\newpage

\section*{Aufgabe 1}
\section*{Aufgabe 2}
\begin{itemize}
    \item[a)] 
	\verb=wait()= unblocks a process, if \verb=S= is larger than 0.\\
	\verb=signal()= increases S by 1\\
	If those two instructions were to be exchanged, the critical section
	would be executed not regarding if there is a queue, since
	the instruction to wait has not been given yet.
    \item[b)]
	If \( S \geq 2 \), Two processes would be allowed to be executed
	at the same time, or, if \( S < 2 \), just one process would
	be started, never giving another process the signal to start.
    \item[c)]
	If \verb=wait()= is missing, the critical section will just be executed
	without checking if there is a queue.\\
	If \verb=signal()= is missing, only S processes can be executed, as
	there is no way to increase S after a process has finished.
\end{itemize}
\section*{Aufgabe 3}
\section*{Aufgabe 4}
The pseudocode can be found in \texttt{pseudo\_4.cpp}
\section*{Aufgabe 5}
\section*{Aufgabe 6}
\begin{itemize}
    \item[a)]
	A Mutex has seperate lock and condition variables. Therefore, condition
	variables have no history. The condition has to be tested seperately,
	instead of relying on a signal.
	Semaphores will remember the Signals given through the Semaphore counter
	\verb=S=. If a Thread broadcasts a signal, the next time another Thread
	calls \verb=wait()=, it will start running immediately, regardless
	of when the signal was given.
    \item[b)]
	The Mutex has to hold the lock, in order for processes not to get stuck
	waiting. This can happen, when a wait function is called first, then a
	signal runs between the time, where the wait checked for a signal, and
	the condition. The Thread will not see that a signal has been called,
	and wait forever.
\end{itemize}

\section*{Aufgabe 7}
\begin{itemize}
    \item 
\begin{table}[H]
    \centering
    \begin{tabular}{lcc}
	\toprule
	Memory & Long term memory & Working memory \\
	Task & Stores fundamental concepts & Stores currently needed things \\ \midrule
	Comparison & Random Access Memory (RAM) & Cache \\
	Speed & Slow & Fast \\
	Volatility & Long storage duration & short storage duration \\
	Unit & Junks / Items & Junks / Items \\
	Size & \( 10 ^{9} Bytes\) & \( 10 ^{2} Bytes \)
	\bottomrule
    \end{tabular}
\end{table}
\item Memory can be moved from the working memory into the long term memory by
    repeating the information several times. This works especially well for
    memorizing vocabulary. Studying everything on one day is generally less efficient
    then repeating the vocabulary several times during the week.
\end{itemize}

\end{document}

