\documentclass{article}

% Deactivate sectsty warning when loading sectsty {{{
\usepackage[immediate]{silence}
\WarningFilter[temp]{latex}{Command}
\usepackage{sectsty}
    \sectionfont{\normalfont\sffamily\bfseries\color{blue!40!black}}
    \subsectionfont{\normalfont\sffamily\bfseries\color{blue!30!black}}
\DeactivateWarningFilters[temp]
\makeatletter % disable the runtime redefinitions
\let\SS@makeulinesect\relax
\let\SS@makeulinepartchap\relax
\makeatother
% }}}

\usepackage[margin=4cm]{geometry}
    \setlength\parindent{0pt}
\usepackage{fancyhdr}
    \pagestyle{fancy}
\usepackage{fontspec}
    \setsansfont{Linux Biolinum O}
\usepackage{polyglossia}
    \setmainlanguage{english}
\usepackage{sectsty}
    \sectionfont{\normalfont\sffamily\bfseries\color{blue!40!black}}
    \subsectionfont{\normalfont\sffamily\bfseries\color{blue!30!black}}
\usepackage{pdfpages}
\usepackage{amsmath}
\usepackage{amssymb}
\usepackage{siunitx}
\usepackage{float}
\usepackage{booktabs}
\usepackage{subcaption}
\usepackage{graphicx}
\usepackage{xcolor}
\usepackage{listings}
    \lstset{language=C++,
	basicstyle=\footnotesize\ttfamily,
	breaklines=true,
	framextopmargin=50pt,
	frame=bottomline,
	backgroundcolor=\color{white!86!black},
	commentstyle=\color{blue},
	keywordstyle=\color{red},
	stringstyle=\color{orange!80!black}}
\usepackage{tikz}

\title{\textsf{\color{blue!40!black}3. Übung IBN}}
\author{Maurice Donner \and Ise Glade}

\begin{document}

\maketitle
\newpage

\section*{Aufgabe 1}
Using spinlocks on a single core unit wouldn't make much sense, as a
critical section cannot be accesed by two processes at the same time.
Therefore there are no race conditions. Locking the only available
processor on a single core (remember we are talking about
spinlocks) might not be the best idea either.
\section*{Aufgabe 2}
\begin{itemize}
    \item[a)] 
	\verb=wait()= unblocks a process, if \verb=S= is larger than 0.\\
	\verb=signal()= increases S by 1\\
	If those two instructions were to be exchanged, the critical section
	would be executed not regarding if there is a queue, since
	the instruction to wait has not been given yet.
    \item[b)]
	If \( S \geq 2 \), Two processes would be allowed to be executed
	at the same time, or, if \( S < 2 \), just one process would
	be started, never giving another process the signal to start.
    \item[c)]
	If \verb=wait()= is missing, the critical section will just be executed
	without checking if there is a queue.\\
	If \verb=signal()= is missing, only S processes can be executed, as
	there is no way to increase S after a process has finished.
\end{itemize}
\section*{Aufgabe 3}
\textbf{a)} The Assertion can be entered by executing the code of Thread A
twice, then executing the code of Thread B, before executing Thread A again.
This way, the condition \texttt{i == 5} is fulfilled, and the error
statement can be executed. Assuming \texttt{Debug\_Assert(false)} is some
piece of code that shall not be executed, this is unwanted, and will only
happen if the threads run in a specific order.\\

\textbf{b)} The \texttt{ManualResetEventSlim} is an object with a single boolean
flag and 3 methods: \texttt{Wait()}, \texttt{Set()} and \texttt{Reset()}.
When a thread calls \texttt{Wait()}, it can only continue, once the reset event
is signaled. This way one can implement the dependency of one thread to another,
by only giving the first \texttt{Set()} signal when a certain part of the program
has been run:
\begin{lstlisting}
Thread A			| Thread B
while (true) {			| while (true) {
    sync.Wait();		| sync.Reset()
    // Execute Second part	| // Execute Second part
}				| sync.Set() }
\end{lstlisting}
To access the \texttt{Debug\_Assert()} statement, we can use the fact, that thread
B doesn't have a \texttt{Wait()} condition. After setting the reset event for the
first time (\texttt{i=2}), the critical region can be entered by thread A,
and then immediately stopped before the \texttt{if}-condition is reached. Thread
B can then increment the counter once again, so that the condition \texttt{i\%2==1}
is fulfilled. If we switch now back to thread A, the \texttt{Debug\_Assert} can be
executed. \\

\textbf{c)} This level can be solved by entering the critical region with thread A,
then releasing the semaphore again with thread B. This shouldn't happen, because
while thread A is inside of the critical region, no other process should be allowed
to release the semaphore again. This is clearly not the case as thread B is allowed
to release the semaphore. \\ The implementation from the lecture looked like this: \\
\begin{minipage}{.49\textwidth}
\begin{lstlisting}
wait (semaphore *S) {
    S->count--;
    if (S->count < 0) {
	//add this process to S->list;
	block(); 
    }
}
\end{lstlisting}
\end{minipage}
\begin{minipage}{.49\textwidth}
\begin{lstlisting}
signal (semaphore *S) {
    S->count++;
    if (S->count <= 0 ) {
	// get and remove
	// process P from S->list;
	wakeup(P);
    }
}
\end{lstlisting}
\end{minipage} \\
In C\#, these routines are exchanged with \texttt{semaphore.Wait()} and
\texttt{semaphore.Release()}, respectively.
Semaphores are defined by \texttt{semaphore = new SemaphoreSlim(a,n)}, where
\texttt{a} is the starting value of the semaphore and \texttt{n} is the maximum number of
allowed processes to enter the lock at the same time.
By calling \texttt{semaphore.Release(n)} the semaphore counter can be increased by
\texttt{n}, allowing the same number \texttt{n} of processes to enter. \\

\textbf{d)} The goal of this level is to raise an exception, which happens
when thread A tries to read a value from the queue, while it is empty.
This can quickly be achieved by releasing the semaphore with thread B,
then interrupting it before it can add an element to the queue.\\
What's abstracted is the internal operations of the queue.
Instead of having to write into the buffer and then having to increment
the consumber pointerm, all while taking care of the maximum
buffersize, you only enqueue and dequeue items.\\

\textbf{e)} Thread 2 is responsible for adding items to the queue and
also signaling all threads to wake up. We can abuse this, by letting all
threads execute until they reach their \texttt{Wait()}-condition. We then
add an item to the queue (thread 2) and then signal all threads to wake up.
If now thread 0 finishes its routine to dequeue an item, it exits the
monitor with \texttt{MonitorExit()}, and thread 1 can enter, and try
to dequeue. This however will raise an exception, since the queue is empty.

\section*{Aufgabe 4}
The pseudocode can be found in \texttt{pseudo\_4.cpp} \\
\section*{Aufgabe 5}
\section*{Aufgabe 6}
\begin{itemize}
    \item[a)]
	A Mutex has seperate lock and condition variables. Therefore, condition
	variables have no history. The condition has to be tested seperately,
	instead of relying on a signal.
	Semaphores will remember the Signals given through the Semaphore counter
	\verb=S=. If a Thread broadcasts a signal, the next time another Thread
	calls \verb=wait()=, it will start running immediately, regardless
	of when the signal was given.
    \item[b)]
	The Mutex has to hold the lock, in order for processes not to get stuck
	waiting. This can happen, when a wait function is called first, then a
	signal runs between the time, where the wait checked for a signal, and
	the condition. The Thread will not see that a signal has been called,
	and wait forever.
\end{itemize}

\section*{Aufgabe 7}
\begin{itemize}
    \item 
\begin{table}[H]
    \centering
    \begin{tabular}{lcc}
	\toprule
	Memory & Long term memory & Working memory \\
	Task & Stores fundamental concepts & Stores currently needed things \\ \midrule
	Comparison & Random Access Memory (RAM) & Cache \\
	Speed & Slow & Fast \\
	Volatility & Long storage duration & short storage duration \\
	Unit & Junks / Items & Junks / Items \\
	Size & \( 10 ^{9} Bytes\) & \( 10 ^{2} Bytes \) \\
	\bottomrule
    \end{tabular}
\end{table}
\item Memory can be moved from the working memory into the long term memory by
    repeating the information several times. This works especially well for
    memorizing vocabulary. Studying everything on one day is generally less efficient
    then repeating the vocabulary several times during the week.
\end{itemize}

\end{document}

