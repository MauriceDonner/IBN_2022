
\documentclass{article}

% Deactivate sectsty warning when loading sectsty {{{
\usepackage[immediate]{silence}
\WarningFilter[temp]{latex}{Command}
\usepackage{sectsty}
    \sectionfont{\normalfont\sffamily\bfseries\color{blue!40!black}}
    \subsectionfont{\normalfont\sffamily\bfseries\color{blue!30!black}}
\DeactivateWarningFilters[temp]
\makeatletter % disable the runtime redefinitions
\let\SS@makeulinesect\relax
\let\SS@makeulinepartchap\relax
\makeatother
% }}}

\usepackage[margin=4cm]{geometry}
    \setlength\parindent{0pt}
\usepackage{fancyhdr}
    \pagestyle{fancy}
\usepackage{fontspec}
    \setsansfont{Linux Biolinum O}
\usepackage{polyglossia}
    \setmainlanguage{english}
\usepackage{sectsty}
    \sectionfont{\normalfont\sffamily\bfseries\color{blue!40!black}}
    \subsectionfont{\normalfont\sffamily\bfseries\color{blue!30!black}}
\usepackage{amsmath}
\usepackage{amssymb}
\usepackage{siunitx}
\usepackage{float}
\usepackage{booktabs}
\usepackage{subcaption}
\usepackage{graphicx}
\usepackage{xcolor}
\usepackage{listings}
    \lstset{language=C++,
	basicstyle=\footnotesize\ttfamily,
	breaklines=true,
	framextopmargin=50pt,
	frame=bottomline,
	backgroundcolor=\color{white!86!black},
	commentstyle=\color{blue},
	keywordstyle=\color{red},
	stringstyle=\color{orange!80!black}}
\usepackage{tikz}

\title{\textsf{\color{blue!40!black}4. Übung IBN}}
\author{Maurice Donner \and Ise Glade}

\begin{document}

\maketitle

\section*{Aufgabe 3}
The \texttt{???} have to be replaced the following way:
\begin{lstlisting}
#include <stdio.h>
#include <sys/types.h>
#include <sys/ipc.h>
#include <sys/shm.h>
#include <sys/wait.h>
#include <stdlib.h>

int main(){

    int i, shmID, *shared_mem, count=0, total=0,rnd;
    shmID = shmgetIPC_PRIVATE, sizeof(int), IPC_CREAT | 0644);
    shared_mem = (int*)shmat(shmID,0,0);
    *shared_mem = 0;
    if (fork())
	for (i=0; i<500; i++){
	    *shared_mem+=1;
	    printf("\n Elternprozess: %i", *shared_mem);
	    sleep(2);
	}
    else
	for (i=0; i<500;i++){
	    *shared_mem+=1;
	    printf("\n Kindprozess: %i", *shared_mem);
	    rnd=rand();
	    sleep(rnd%3);
	}
    shmdt(shared_mem);
    shmctl(shmID, IPC_RMID, 0);
    return 0;
}
\end{lstlisting}

\begin{itemize}
    \item \texttt{shmget} creates a segment of shared memory
    \item If a process wants to use this segment, it has to attach to it with
	\texttt{shmat}
    \item Once the process is finished, it detaches from the shared memory with
	\texttt{shmdt}
    \item \texttt{shmctl} performs a specified operation on the shared memory segment
\end{itemize}

The shared counting doesn't seem to be determinstic. This is because of race conditions
concerning the shared variable \texttt{shared\_mem}

\section*{Aufgabe 8}
\textbf{a)} 
\begin{table}[H]
    \centering
    \begin{tabular}{lcr}
	\centering
	Matrikelnummer & $\rightarrow$ & Seitennummer (page number)\\
	Wohnadresse & $\rightarrow$ & Rahmennummer (frame number)\\
	Verzeichnis & $\rightarrow$ & Seitentabelle (page table)\\
    \end{tabular}
\end{table}

\textbf{b)} The table would contain entries equal to the number of available
matriulation numbers. With 7 digits from 0 to 9, this would make \( 10 ^{7} \) entries.
This number could be reduced, if one considers the maximum number of students enroled
at a single point in time. This way, matriculation numbers could be repurposed
if a student is exmatriculated for any reason such as submitting a sample solution
for his exercise sheet.

\section*{Aufgabe 9}
Assuming the counting of addresses starts at 0: (syntax: pagenumber | offset)\\
\[ 
    2456 \ \hat = \ 2 | 408 \quad 16382 \ \hat = \ 15 | 1022 \quad 30000 \ \hat =
    \ 29 | 304 \quad 4385 \ \hat = \ 4 | 289
\]
In C, a calculation could look like this:
\begin{lstlisting}
int page_number = value/pagesize // implicit conversion to int
int offset = value%pagesize
\end{lstlisting}



\end{document}

