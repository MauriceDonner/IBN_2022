
\documentclass{article}

% Deactivate sectsty warning when loading sectsty {{{
\usepackage[immediate]{silence}
\WarningFilter[temp]{latex}{Command}
\usepackage{sectsty}
    \sectionfont{\normalfont\sffamily\bfseries\color{blue!40!black}}
    \subsectionfont{\normalfont\sffamily\bfseries\color{blue!30!black}}
\DeactivateWarningFilters[temp]
\makeatletter % disable the runtime redefinitions
\let\SS@makeulinesect\relax
\let\SS@makeulinepartchap\relax
\makeatother
% }}}

\usepackage[margin=4cm]{geometry}
    \setlength\parindent{0pt}
\usepackage{fancyhdr}
    \pagestyle{fancy}
\usepackage{fontspec}
    \setsansfont{Linux Biolinum O}
\usepackage{polyglossia}
    \setmainlanguage{english}
\usepackage{sectsty}
    \sectionfont{\normalfont\sffamily\bfseries\color{blue!40!black}}
    \subsectionfont{\normalfont\sffamily\bfseries\color{blue!30!black}}
\usepackage{amsmath}
\usepackage{amssymb}
\usepackage{siunitx}
\usepackage{float}
\usepackage{booktabs}
\usepackage{subcaption}
\usepackage{graphicx}
\usepackage{xcolor}
\usepackage{listings}
    \lstset{language=Python,
	basicstyle=\footnotesize\ttfamily,
	breaklines=true,
	framextopmargin=50pt,
	frame=bottomline,
	backgroundcolor=\color{white!86!black},
	commentstyle=\color{blue},
	keywordstyle=\color{red},
	stringstyle=\color{orange!80!black}}
\usepackage{tikz}

\title{\textsf{\color{blue!40!black}6. Übung IBN}}
\author{Maurice Donner \and Ise Glade}

\begin{document}

\section{Aufgabe 6}
The Programm is written in \texttt{C++}. The clock algorithm
is implemented the following way:\\
\begin{itemize}
    \item A \texttt{page} struct, that contains page number and an R-Bit
\begin{lstlisting}
struct page {
    char pnumber;
    bool R;
}
\end{lstlisting}
    \item A Ringbuffer, implemented by a simple \texttt{page}-Array

Output for Reference A:

\begin{lstlisting}
Anzahl der Seitenrahmen (default 3): 3
Referenzfolge (default 70120304230321201701): 
7 -> [(7), 10 , 10 ]
0 -> [ 7 ,(0), 10 ]
1 -> [ 7 , 0 ,(1)]
2 -> [(2), 0 , 1 ]
0 -> [ 2 , 0 , 1 ]
3 -> [ 2 , 0 ,(3)]
0 -> [ 2 , 0 , 3 ]
4 -> [(4), 0 , 3 ]
2 -> [ 4 , 0 ,(2)]
3 -> [(3), 0 , 2 ]
0 -> [ 3 , 0 , 2 ]
3 -> [ 3 , 0 , 2 ]
2 -> [ 3 , 0 , 2 ]
1 -> [ 3 ,(1), 2 ]
2 -> [ 3 , 1 , 2 ]
0 -> [(0), 1 , 2 ]
1 -> [ 0 , 1 , 2 ]
7 -> [ 0 , 1 ,(7)]
0 -> [ 0 , 1 , 7 ]
1 -> [ 0 , 1 , 7 ]
\end{lstlisting}

\end{itemize}

\end{document}

